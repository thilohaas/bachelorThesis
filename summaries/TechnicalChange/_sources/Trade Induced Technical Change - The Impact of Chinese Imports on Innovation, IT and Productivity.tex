\documentclass[12pt,a4paper,fleqn]{article}

%% Font
\usepackage{mathpazo}

%% Mathematics
\usepackage{amsmath, latexsym}

%% Graphics
\usepackage{graphicx}

%% Line spacing
\usepackage[onehalfspacing]{setspace}

%% Bibliography
\usepackage{natbib}

\begin{document}

\begin{center}
{\bf \large Trade Induced Technical Change - The Impact of Chinese Imports on Innovation, IT and Productivity, Bloom, Draca, Reenen 2011}\\[1cm]
NBER Working Paper Series 16717\\
January 2011\\
JEL F14,L25,L60,O33\\[2cm]

\begin{tabular}{rl}
\hline
Data Source & ?�\\
Data Time & 1996-2007\\
Country & 12 European Countries \\
Focus & Imports from low-wage countries (china)\\
& have significant effects (within and \\
& between firms) on innovation\\

\hline
\end{tabular}

\end{center}

\newpage
%%%%%%%%%%%%%%%%Abstract%%%%%%%%%%%%%%%%%%%
\section{Abstract}
Examine impact of Chinese import competition on patenting, IT, R\&D and TFP\\
Panel data of 500k firms between 1996-2007 across twelve European countries\\
Mainly two effects:\\
\begin{itemize}
	\item led to increased innovation within firms
	\item reallocated employment towards more innovative and technologically advanced firms
\end{itemize}
both effects account for around 15\% of European technology upgrading and falls in employment, profits, prices and the skill share\\
Import competition from developed countries had no effect on innovation\\

\section{Introduction}
we argue that increased Chinese trade has also induced faster technical change from both innovation and the adoption of new technologies, contributing to productivity growth\\
exist several case studies:\\
\begin{itemize}
	\item Bartel, Ichinowski and Shaw (2007) - American valve-makers
	\item Freeman and Kleiner (2005) - footwear
	\item Bugamelli, Schivardi and Zizza (2008) - italian manufacturer
\end{itemize}
countribution of paper: confirm the importance of low wage country trade for technical change\\
\\
rise of emerging economies (china, india, mexico,..) coincided with increase in wage inequality - many authors seen a link between the two trends and accounted the effect of trade less importantant.
importance of trade:
\begin{itemize}
	\item most of the studies used date before 1990s which is before the rise of china
	\item didnt take the impact of trade through offshoring rather than final goods into account
	\item trade can also affect the incentives to adopt new technologies and therefore stimulate technical progress
\end{itemize}
Two core results:\\
\begin{itemize}
	\item Chinese import competition increases innovation and TFP within surviving firms.
	\item Chinese import competition reduces employment and survival probabilities in low-tech firms 
	\item Chinese imports significantly reduce prices, profitability and the demand for unskilled workers as basic theory would suggest.
	\item Imports from developed countries appear to have no impact on technology.
\end{itemize}
model, further developed in Bloom, Romer and Van Reenen (2010), that explains how trade from China drives innovation in exposed firms. relies on trapped-factors (factors of production which are costly to move between firms) $\rightarrow$ Hence, by reducing the profitability of current low-tech products and freeing up inputs to innovate and produce new products, Chinese trade reduces the opportunity cost of innovation.\\
\\
find empirical support for 2 model predictions:
\begin{itemize}
	\item import competition from low wage countries like China has a greater effect on innovation than imports from high wage countries.
	\item firms with more trapped factors will respond more strongly to import threats.
\end{itemize}

\section{Conclusions}
identify the trade-induced technical change hypothesis\\
Results:
\begin{itemize}
	\item TFP and absolute levels of patenting, R\&D and IT have risen in firms who were more exposed to increases in Chinese imports (the within firm effect)
	\item in sectors more exposed to Chinese imports, jobs and survival rate fell in low- tech firms (measured by indicators such as IT and patenting intensity), but are relatively protected in high-tech firms (the between firm effect
\end{itemize}
Both within and between firm effects generate technological upgrading.\\
China could account for around 15\% of the overall technical change in Europe between 2000 and 2007.\\
\\
{\bf our results imply that reducing import barriers against low wage countries like China may bring important welfare gains through technical change, subject to the caveats over equilibrium effects discussed in sub-section VI.D.}
	

\end{document}