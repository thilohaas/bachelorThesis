\documentclass[12pt,a4paper,fleqn]{article}

%% Font
\usepackage{mathpazo}

%% Mathematics
\usepackage{amsmath, latexsym}

%% Graphics
\usepackage{graphicx}

%% Line spacing
\usepackage[onehalfspacing]{setspace}

%% Bibliography
\usepackage{natbib}

\begin{document}

\begin{center}
{\bf \large Skills, Tasks and Technologies: Implications for Employment and Earnings, Acemoglu, Autor 2010}\\[1cm]
NBER Working Paper Series 16082\\
June 2010\\
JEL J20, J23, J24, J30, J31, O31, O33\\[2cm]

\begin{tabular}{rl}
\hline
Data Source & -�\\
Data Time & -\\
Country & general theoretical model \\
Focus & task based framework proposition to extend\\
& the cannonical model\\
\hline
\end{tabular}

\end{center}

\newpage
%%%%%%%%%%%%%%%%Abstract%%%%%%%%%%%%%%%%%%%
\section{Introduction}
canonical model:\\
two-skill groups performing two distinct and imperfectly substitutable occupations\\
Technology is assumed to take a factor-augmenting form; Changes in this factor-augmenting technology then capture skill biased technical change\\
\\
modern labor markets and recent empirical trends necessitates a richer framework.\\
\\
two shortcomings of the canonical model\\
\begin{itemize}
	\item does not include a meaningful role for �tasks�; imposes a one-to-one mapping between skills and tasks.\\
		we suggest following Autor, Levy and Murnane (2003)
	\item it treats technology as exogenous and assumes that technical change is, by its nature, skill biased\\
		but extent of skill bias of technical change has varied over time and across countries\\
		Acemoglu (1998, 2002a) suggested that the endogenous response of technology to labor market conditions
\end{itemize}
we show that:
\begin{itemize}
	\item low skill workers experienced significant real earnings declines 
	\item non-monotone changes in earnings levels across the earnings distribution
	\item non-monotone shifts in the composition of employment
	\item �polarization� of employment reflect also a change in the allocation of skill groups across occupations
	\item technological developments and recent trends in o�shoring and outsourcing appear to have directly replaced workers in certain occupations and tasks
\end{itemize}
we propose a task based framework\\
three types of skills�low, medium and high\\
Workers have di�erent comparative advantages\\
similar to Ricardian trade models. Given the prices of (the services of) di�erent tasks and the wages for di�erent types of skills in the market, firms (equivalently, workers) choose the optimal allocation of skills to tasks\\
Importantly, the model allows for new technologies that may directly replace workers in certain tasks\\
The canonical model is in fact a special case of this more general task-based model,\\
richer implications because of the endogenously changing allocation of skills to tasks.\\

\section{3 The Cannonical Model}
high skill worker ratio determined by relative supply and relative demand for skills\\
relative demand for skills increases over time due to skill biased technology (increased demand / complementarity)\\
=> Tinbergen�s famous race between technology and supply of skills\\
\subsection{3.1}
model:\\
high / low skill workers\\
competitive labor markets\\
constant elasticity of substitution aggregate production function\\
\\
elasticity of substitution between high and low skill determines how changes in relative supply affect skill premia\\
\\
Total supply of Low and High skill labor in the economy\\
$L = \int_{i \in L}   l_i  \mathrm{d} i$ and $H = \int_{i \in H} \!  h_i  \mathrm{d} i$\\
\\
Production function for aggregate economy\\
$$Y = [(A_{L} L)^{\frac{\sigma - 1}{\sigma}} + (A_{H} H)^{\frac{\sigma - 1}{\sigma}}]^{\frac{\sigma}{\sigma - 1}}$$
$A_L$ / $A_H$ factor-augmenting technology terms\\
high/low skill workers are gross substitutes for $\sigma > 1$\\
$\sigma = 1$ cobb-douglas production function\\
$\sigma \rightarrow 0$ leontief production function\\
$\sigma \rightarrow \infty$ perfect substitutes\\
\\
Technologies are factor augmenting $\rightarrow$ no explicitly skill replacing technologies\\
\\
due to competitive labor market:\\
$$\omega_L = \frac{\partial Y}{\partial L} = A_L^{\frac{\sigma - 1}{\sigma}} [A_L^{\frac{\sigma - 1}{\sigma}} + A_H^{\frac{\sigma - 1}{\sigma}}(H/L)^{\frac{\sigma - 1}{\sigma}}]^{\frac{1}{\sigma - 1}}$$
higher fraction of H/L leads to a low-skill wage increase\\
factor augmenting technical change (higher $A_L/A_H$) leads to a low-skill wage increase\\
\\
skill premium\\
$$\omega = \frac{\omega_H}{\omega_L} = (\frac{A_H}{A_L})^{\frac{\sigma - 1}{\sigma}} (\frac{H}{L})^{\frac{-1}{\sigma}}$$
.
2 Forces in Tinbergen's race:\\
\begin{itemize}
	\item relative high-skill biased technology\\
		$\frac{\partial ln\omega}{\partial ln(A_H/A_L)} = \frac{\sigma - 1}{\sigma}$\\
		changes in technology increasing the demand of skills\\
		$\sigma > 1$ relative improvements in high skill augmenting technology will increase skill premium
	\item relative supply of high skill labor\\
		$\frac{\partial ln\omega}{\partial ln H/L} = - \frac{1}{\sigma} < 0$\\
		for given skill bias of technology, an increase in relative supply of skills reduces the skill premium with an elasticity of $1/\sigma$\\
		This means if $A_L/A_H$ stayed roughly the same a high increase in the relative skill supply should have lowered the wage premium significantly which was not the case in the recent decades
\end{itemize}
In studies with college/non-college workers o is estimated between 1.4 and 2\\
\\

\subsection{3.2}
assuming log lin increase in demand for skills coming from technology:\\
$ln(\frac{A_{H,t}}{A_{L,t}}) = \gamma_0 + \gamma_1 t$ will lead to $ln\omega_t = \frac{\sigma -1}{\sigma}\gamma_0 + \frac{\sigma -1}{\sigma}\gamma_1 t - \frac{1}{\sigma} ln(\frac{H_t}{L_t})$\\
technological development takes place at a constant rate while supply of skills may vary\\
when H/L grows slower than the rate of skill biased technical change, (o-1)y1, the skill premium will increase\\
\\

\subsection{6 conclusions}
task-based framework a unique final good is produced combining services of a continuum of tasks.\\
$$ Y = exp[\int_0^1 ln y(i) \mathrm{d}i] $$
productivity of tasks:
$$ y(i) = A_L \alpha_L(i)l(i) + A_M\alpha_M(i)m(i) + A_H\alpha_H(i)h(i) + A_K\alpha_K(i)k(i)$$
worker has one of three types of skills: low, medium and high\\
pattern of comparative advantage\\
equilibrium allocation of skills to tasks is determined by two thresholds, $I_L$ and $I_H$, such that all tasks below the lower threshold ($I_L$) are performed by low skill workers, all tasks above the higher threshold ($I_H$) are performed by high skill workers\\
no arbitrage conditions:\\
$$\frac{A_M\alpha_M(I_H)M}{I_H - I_L} = \frac{A_H\alpha_H(I_H)H}{1- I_H}$$
$$\frac{A_L\alpha_L(I_L)L}{I_L} = \frac{A_M\alpha_M(I_L)M}{I_H- I_L}$$

despite the endogenous allocation of skills to tasks, the model is tractable, and that relative wages among skill groups depend only on relative supplies and the equilibrium threshold tasks\\
technical change favoring one type of worker can reduce the real wages of another group\\
framework enables us to model the introduction of new technologies that directly substitute for tasks previously performed by workers of various skill levels\\

\end{document}