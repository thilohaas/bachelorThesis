\documentclass[12pt,a4paper,fleqn]{article}

%% Font
\usepackage{mathpazo}

%% Mathematics
\usepackage{amsmath, latexsym}

%% Graphics
\usepackage{graphicx}

%% Line spacing
\usepackage[onehalfspacing]{setspace}

%% Bibliography
\usepackage{natbib}

\begin{document}

\begin{center}
{\bf \large Highly-Educated Immigrants and Native Occupational Choice, Peri, Sparber 2010}\\[1cm]
Faculty Scholarship Working Papers Series. Paper 16\\
2010\\
JEL J61, J31, F22\\[2cm]

\begin{tabular}{rl}
\hline
Data Source &�O*NET\\
Data Time & 1950 - 2007\\
Country & US \\
Focus & substitutability between highly-educated native\\
& and foreign workers\\
\hline
\end{tabular}

\end{center}

\newpage
%%%%%%%%%%%%%%%%Abstract%%%%%%%%%%%%%%%%%%%
\section{Abstract}
assess whether native-born workers with graduate degrees respond to an increased presence of highly-educated foreign-born workers by choosing new occupations with different skill content.\\
\\
highly-educated native and foreign-born workers are imper- fect substitutes\\
When the foreign-born pro- portion of highly-educated employment within an occupation rises, native employees with graduate degrees choose new occupations with less analytical and more communicative content.\\

\section{Introduction}
Foreign-born share of highly educated employees in US rose between 1950-2007 from 5.9\% to 18.1\%\\

\subsection{existing literature}
\begin{itemize}
	\item substitutability of highly-educated native and immigrant workers:\\
		perfectly substitutable:\\
		\begin{itemize}
			\item Borjas (2003, 2006)
			\item Borjas and Katz (2005)
		\end{itemize}
	\item focusing on less-educated: \\
		imperfect substitutable (posess different skills):\\
		\begin{itemize}
			\item Manacorda et. al. (2006)
			\item Ottaviano and Peri (2008)
			\item Peri and Sparber (2009)
		\end{itemize}
\end{itemize}

\subsection{substitutability}
Imperfect subsitutability allows natives to specialize in occupations where they have a comparative advantage to mitigate possible wage losses from immigration.\\
\\
assumption: highly-educated native and foreign workers provide two general skills
\begin{itemize}
	\item interactive (communication) tasks
	\item quantitative (analytical) tasks
\end{itemize}

highly-educated immigrants, relative to native-born workers, will have imperfect language skills, knowledge of local networks, and familiarity with social norms.\\
comparative advantage of natives: communication skills\\
comparative advantage of foreigners: analytical skills\\

\subsection{data}
O*NET \& Current Population Survey (CPS) from 2003-2008 $\rightarrow$ measuring the skills of native-born workers with graduate degrees\\
Census Data (1990) \& American Community Surveys (2002-2007) $\rightarrow$ share of highly-educated employment\\


\subsection{results}
native and foreign-born with graduate degrees are imperfect substitutes.\\
\\
individual native workers move to occupations with higher communication content in response to an increase in the share of immigrants within their original occupation\\
\\
little to no evidence that highly- educated native employees in occupations with large increases in the proportion of similarly- educated immigrants are more likely to become unemployed or leave the labor force


\end{document}