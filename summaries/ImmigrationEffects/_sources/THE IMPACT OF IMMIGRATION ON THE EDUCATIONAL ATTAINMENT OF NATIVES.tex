\documentclass[12pt,a4paper,fleqn]{article}

%% Font
\usepackage{mathpazo}

%% Mathematics
\usepackage{amsmath, latexsym}

%% Graphics
\usepackage{graphicx}

%% Line spacing
\usepackage[onehalfspacing]{setspace}

%% Bibliography
\usepackage{natbib}

\begin{document}

\begin{center}
{\bf \large The Impact of Immigration on the Educational Attainment of Natives Jennifer Hunt}\\[1cm]
NBER Working Paper No. 18047\\
May 2012\\
JEL No. J15\\[2cm]

\begin{tabular}{rl}
\hline
Data Source &�census data\\
Data Time & 1940-2010\\
Country & US \\
Focus & Impact of immigration on natives completion of 12 years of \\
& schooling comparing results across ethnicity, race and gender \\
\hline
\end{tabular}

\end{center}

\newpage
%%%%%%%%%%%%%%%%Abstract%%%%%%%%%%%%%%%%%%%
\section{abstract}

An increase of {\bf one percentage point} in the share of immigrants in the population aged 11-64 increases the probability that natives aged 11-17 eventually complete 12 years of schooling by {\bf 0.3 percentage points}, and increases the probability for native-born blacks by {\bf 0.4 percentage points.}\\
\\
two possible effects:\\
\begin{itemize}
	\item rising low skill immigration could reduce native high school graduation rates 
		$\rightarrow$ reduce quality of k-12 education
		\begin{itemize}
			\item non-native speaking immigrants may lower the pace of instruction in some subjects at school \& teacher 
			lowers expectations for all students $\rightarrow$ lowering education quality
			\item lower educational quality for natives will lower return to education and therefore may induce natives to complete fewer years of school $\rightarrow$ lowering education quality
		\end{itemize}
		
	\item low skill immigration-induced changes in labor market incentives for educational attainmend 
		$\rightarrow$ increase quality of k-12 education
		\begin{itemize}
			\item wealthy parents move their children to better schools / schools with fewer immigrants
			\item ...
		\end{itemize}
\end{itemize}

Papers who examined impact of immigrants on native test scores in europe and israel:\\
Brunello and Rocco (2011), Geay et al. (2012), Gould et al. (2009), Jensen and Rasmussen (2011), Ohinata and van Ours (2011).\\
\\
why wages of high school dropouts decline so little when facing immigration:
\begin{itemize}
\item Peri and Sparber (2009): unskilled natives exploit their comparative advantage to avoid competition with immigrants, by shifting to mor communication-intensive occupations.
\item education upgrading (betts, 1998) one percentage point increase in the share of immigrants with less than 12 years school in the population aged 18-64 increases the eventual native completion rate by 0.8 percentage points, with larger effects for native-born minorities
\end{itemize}

\section{Data and descriptive statistics}

%\newpage
\section{Conclusion}
natives� probability of completing 12 years of education is increased by immigration\\
\\
education upgrading is prompted by a higher return to high school due to immigration of high school dropouts, I find that natives� probability of completing 12 years of education is increased by greater presence of adult immigrants with less than 12 years of education.\\
\\
While immigrants age 11�17 have no effect on the 12�year completion rates of non� Hispanic white natives, moderate negative effects on the completion rates of black natives cannot be excluded.\\
\\
children of parents with less than 12 years of education have a deleterious impact on native completion rates, while children with a parent with 12 or more years of education have a positive effect\\
\\
native Hispanic males are most sensitive to child immigration\\
\\
may be an indicator that native students are most affected in school when exposed to culturally similar immigrants



\end{document}