\documentclass[12pt,a4paper,fleqn]{article}

%% Font
\usepackage{mathpazo}

%% Mathematics
\usepackage{amsmath, latexsym}

%% Graphics
\usepackage{graphicx}

%% Line spacing
\usepackage[onehalfspacing]{setspace}

%% Bibliography
\usepackage{natbib}

\begin{document}

\begin{center}
{\bf \large Immigration and Occupations in Europe, D'Amuri, Peri 2010}\\[1cm]
CDP No 26/10\\
Centre for Research and Analysis of Migration Department of Economics, University College London\\
2010\\
JEL J24, J31, J61\\[2cm]

\begin{tabular}{rl}
\hline
Data Source &�?\\
Data Time & 1996-2007\\
Country & Western Europe \\
Focus & identify whether immigration has also been a force\\
& that promoted the specialization of native workers in\\
& Europe towards abstract-complex occupations and\\
& away from manual-routine ones\\
\hline
\end{tabular}

\end{center}

\newpage
%%%%%%%%%%%%%%%%Abstract%%%%%%%%%%%%%%%%%%%
\section{Abstract}
Effect of immigrants on natives� job specialization in Western Europe.\\
Immigrants take more manual-routine type of occupations and push natives towards more abstract-complex jobs\\
reallocation is larger in countries with more flexible labor laws\\
Immigration does not change much the relative compensation of the two types of tasks but it promotes the specialization of natives into the first type\\

\section{Introduction}
Observation: increase in demand for jobs requiring complex and abstract skills coupled with a decrease in the demand for manual-routine\\
These tendencies have been documented by:
\begin{itemize}
	\item US (Acemoglu and Autor, 2010)
	\item European Countries (Goos et al., 2009)
\end{itemize}
these are the main contributors to changing the aggregate de- mand for specific jobs in rich countries. this paper explores another dimension that may have produced a shift in the supply of tasks in rich countries:\\
goal of this paper is to identify whether immigration has also been a force that promoted the specialization of native workers in Europe towards abstract-complex occupations and away from manual-routine ones\\
\\
To check whether specialization is actually caused by the inflow of immigrants: instrument inspired by Card(2001).\\
Hence, the predicted inflow of immigrants, based on their initial shares, is a valid instrument for their actual in- flow.\\
control proxies for other processes that are moving natives towards complex-abstract tasks: may be country or skill-specific, namely technological change and trade.\\
\\
Aggregate European data contain patterns consistent with the idea that immigrants
and natives specialize in di??erent production tasks and such specialization increased over time.\\
While the average native worker increasingly specialized in Complex production tasks (as revealed by their occupational distribution) the average immigrant worker experienced, if anything, the opposite trend.\\
\\
{\bf an increase in 10 percentage points in the share of migrants on total population is associated with an increase of 4 points in relative Complex/Non Complex task intensity}\\
\\
as immigrants take manual-routine jobs, native move towards Complex-Abstract tasks for which they have comparative advantages\\

\section{4.2 Immigrants and native specialization}

$$ln(C_D)_{j,c,t} = \gamma_C * ln(f_{j,c,t} + d_{j}^{C} + d_{c,t}^{C} + d_{edu,t}^{C} + \epsilon_{j,c,t}^{C} $$

skill, country-year and education-year fixed effects ($d_{j}^{C}, d_{c,t}^{C} $ and $d_{edu,t}^{C}$) \\

positive and significant value of $\gamma_C$ implies that an increase in immigrants in the cell pushes natives to specialize in more complex-cognitive tasks relative to cells with smaller inflows of immigrants\\

estimates very consistent across specifications..\\

First, for almost all the estimates the change in share of immigrants in a cell is associated with an increase in the intensity of Complex tasks\\
\\
stimated elasticity is between 0.035 and 0.074 $\rightarrow$ doubling of the share of immigrants in a cell is associated with an increase in the supply of complex tasks by natives between 3.5 and 7.4\%.\\[.5cm]

same for non-complexe tasks (NC):\\
coefficients of table 3 are either negative or indistinguishable from 0\\
the results imply that natives move on average to occupations with larger content of complex tasks and about the same or a bit smaller content of manual-routine tasks\\




\end{document}