\documentclass[12pt,a4paper,fleqn]{article}

%% Font
\usepackage{mathpazo}

%% Mathematics
\usepackage{amsmath, latexsym}

%% Graphics
\usepackage{graphicx}

%% Line spacing
\usepackage[onehalfspacing]{setspace}

%% Bibliography
\usepackage{natbib}

\begin{document}

\begin{center}
{\bf \large Task Specialization, Immigration, and Wages, Peri \& Sparber}\\[1cm]
Faculty Scholarship Working Paper Series, 16\\
2010\\
JEL J24, J31, J61\\[2cm]

\begin{tabular}{rl}
\hline
Data Source &�O*NET\\
Data Time & 1960-2000\\
Country & US \\
Focus & immigration will cause natives to reallocate their task supply \\
&  foreign-born workers specialize in manual-physical labor skills\\
& natives pursue jobs more intensive in communication-language tasks\\
\hline
\end{tabular}

\end{center}

\newpage
%%%%%%%%%%%%%%%%Abstract%%%%%%%%%%%%%%%%%%%
\section{abstract}

immigrants and natives of comparable educational attainment and experience possess unique skills that lead them to specialize in different occupations, which mitigates natives� wage losses from immigration.\\
focus on workers with little educational ettainment: less educated native and immigrant workers specialize in different production tasks.\\
immigrants are likely to have imperfect language ("communication") but possess physical ("manual") skills similar to native born workers. $\rightarrow$ immigration encourages workers to specialize\\
\\
Therefore, productivity gains from specialization, coupled with the high compensation paid to communication skills, imply that foreign-born workers do not have a large, adverse effect on the wages paid to less educated natives.\\
\\
the positive wage effect of specializing in language-intensive occupations, native-born workers can protect their wages and mitigate losses due to immigra- tion by reallocating their tasks.\\
\\
Altogether, our findings agree in spirit with those of Card (2001), Card and Lewis (2007), Card (2007), and Cortes (2008), while adding a new dimension and more microfoundations to the structural framework introduced by Borjas (2003) and refined by Ottaviano and Peri (2008).\\


\section{Theoretical Model}

\section{Data: Task Variables and Instruments}

\section{Empirical Results}

\section{Simulated Effects of Immigration on Real Wages}

\section{Conclusions}
Effects of immigration on wages paid to native-born workers with low levels of educational attainment depend upon two critical factors:
\begin{itemize}
	\item do immigrants take similar jobs like native workers or different jobs due to inherent comperative advantages
	\item do native respond to immigration and adjust their occupation choices
\end{itemize}

when immigration increases manual task supply, the relative compensation paid to communication skill rises $\rightarrow$ rewarding natives who move to language-intensive jobs\\

States with large immigrant inflows:
\begin{itemize}
\item native workers shifted to communication intensive/less physical skill occupations. 
\item Immigrants more than compensated for the change in skill supply among natives
\end{itemize}
$\Rightarrow$ overall increase in manual task supply \\
$\Rightarrow$ driving communication task-intensive occupations to earn higher wages in those states.\\[0.5cm]

immigration only reduced average real wages paid to less- educated US-born workers by 0.3 percent between 1990 and 2000\\

\end{document}