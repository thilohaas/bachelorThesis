\documentclass[12pt,a4paper,fleqn]{article}

%% Font
\usepackage{mathpazo}

%% Mathematics
\usepackage{amsmath, latexsym}

%% Graphics
\usepackage{graphicx}

%% Line spacing
\usepackage[onehalfspacing]{setspace}

%% Bibliography
\usepackage{natbib}

\begin{document}

\begin{center}
{\bf \large The Effect of immigration on productivity: Evidence from US States, Peri}\\[1cm]
NBER Working Paper, 15507\\
November 2009\\
JEL No. F22,J61,R11\\[2cm]

\begin{tabular}{rl}
\hline
Data Source &�National accounting data \& census data (IPUMS), (O*NET) \\
Data Time & 1960 - 2000, 2006 \\
Country & US \\
Focus & effect of immigration on capital intensity, factor productivity,\\
& skill-bias of aggregate productivity\\
\hline
\end{tabular}

\end{center}

\newpage
%%%%%%%%%%%%%%%%Abstract%%%%%%%%%%%%%%%%%%%
\section{Abstract}
analyzing effect of immigrants on state employment, average hours worked, physical capital accumulation and, most importantly, total factor productivity and its skill bias.\\
using location of a state relative to the mexican border and entrance points (ports) \& existing communities of immigrants before 1960.\\
\\
no evidence that immigrants crowded-out employment and hours worked by natives.\\
robust evidence that immigration increased total factor productivity and decreased capital intensity and the skill-bias of production technologies.\\
\\
Our results suggest that immigrants promoted efficient task specialization, thus increasing TFP and, at the same time, promoted the adoption of unskilled-biased technology as the theory of directed technologial change would predict.\\
{\bf increase in employment in a US state of 1\% due to immigrants produced an increase in income per worker of 0.5\% in that state.}

\newpage
\section{Introduction}
this paper:\\
identify the impact of immigration on capital intensity, total factor productivity and the skill-bias of aggregate productivity using national accounting data combined with census data.\\
The factors that we explicitly control for are the intensity of R\&D and innovation, the adoption of computers, the openness to international trade as measured by the export intensity and the sector-composition of the state\\
\\
it shows that a measure of task-specialization of native workers induced by immigrants explains half to two thirds of the positive productivity effect.\\
\\
results, combined with a constant capital-labor ratio in production suggest that these {\bf productivity gains may arise due to the efficient allocation of skills to tasks, as immigrants are allocated to manual-intensive jobs, promoting competition and pushing natives to perform communication-intensive tasks more efficiently}.

\newpage
\section{The Efficient Task-Specialization Hypothesis}
two mechanisms proposed and studied in the previous literature that can jointly explain the positive productivity effect of immigrants and its skill-bias:
\begin{itemize}
	\item Lewis and Card (2007) markets with increase in less educ. immigrants $\rightarrow$ many sectors show higher intensity of unskilled workers\\
	Lewis (2005) in those markets is a slower adoption of skill-intensive techniques 
	\item Acemoglu (2002) directed technological change \& appropriate technological adoption
\end{itemize}
$\Rightarrow$ availability of unskilled workers pushes firms to adopt technologies that are more efficient and intensive in the use of unskilled workers. (more investment in less-skill intensive tech to increase the return of low skilled labour)\\
\\
(Peri and Sparber 2009) we show that in states with large inflows of immigrants, natives with lower education tend to specialize in more communication-intensive production tasks, leaving to immigrants more manual-intensive tasks.\\
produces increased task-specialization following comparative advantages and results in efficiency gains, especially among less educated workers.\\
\\
regression:\\
idea is that if immigrants affect the efficiency of production in a state, by reallocating natives toward communication tasks and by undertaking manual tasks, leading to an overall productivity improvement, we should observe the productivity effect of immigrants mostly through the task-reallocation of natives\\
\\
{\bf the effect of controlling for the "change in specialization" on the estimated coefficient of immigration on TFP is much more drastic than the effect of introducing any other control. This is a sign that a large part of the positive productivity effect may actually go through the efficient re-allocation of natives and immigrants across production tasks.}

\end{document}