\documentclass[12pt,a4paper,fleqn]{article}

%% Font
\usepackage{mathpazo}

%% Mathematics
\usepackage{amsmath, latexsym}

%% Graphics
\usepackage{graphicx}

%% Line spacing
\usepackage[onehalfspacing]{setspace}

%% Bibliography
\usepackage{natbib}

\begin{document}

\begin{center}
{\bf \large Auswirkungen der Personenfreiz�gigkeit auf den Schweizer Arbeitsmarkt, SECO 2012}\\[1cm]
2012\\[2cm]

\begin{tabular}{rl}
\hline
Data Source &�?\\
Data Time & ?\\
Country & Switzerland \\
Focus & ?\\
\hline
\end{tabular}

\end{center}

\newpage
%%%%%%%%%%%%%%%%Abstract%%%%%%%%%%%%%%%%%%%
\section{Management Summary}
Entsprechend Zielsetzung der schweizerischen Migrationspolitik werden heute auslaendische Arbeitskraefte prioritaer im EU/EFTA-Raum rekrutiert.\\
\\
Am staerksten fiel der Zuwachs der Netto-Zuwanderung in den Kantonen Waadt, Basel-Stadt, Z�rich, Wallis, Zug, Genf und Neuenburg aus.\\
\\
Zuwanderung aus den EU-Staaten reagiert bekanntlich sehr ausgepraegt auf die Nachfrage der Unternehmen nach Arbeitskraeften.\\
Hoechstwert 2008, Finanzkriese 2009, Erholung 2011.

\subsection{Auswirkungen auf den Schweizer Arbeitsmarkt}
Frage ob Zuwanderung allenfalls zu einer Verdraengung der ansaessigen Erwerbspersonen fuehrt.\\
\\
Qualifikationsmix der Zuwanderer hat sich in den letzten Jahren gewandelt: letzen Jahre mehrheitlich mittler bis hoeheres Ausbildungsniveau:\\
2002-2010: 83\% Abschluss auf der Sekundarstufe II (Matura/berufliche Grundbildung) \& 51\% tertiaere Bildungsabschluss (hoehere Berufsbildung, Fachhochschule, Uni).\\
\\
Scheint es, dass die zugewanderten Arbeitskraefte aus der EU in der Mehrzahl eine gute Ergaenzung des ansaessigen Arbeitskraeftepotentials in stark wachsenden Arbeitsmarktsegmenten darstellen.\\
\\
Berufsgruppen mit dem deutlichsten Zuwachs: Fuehrungskraefte, akademische Berufe, Techniker/innen und gleichrangige Berufe\\
\\
Lohnstruktur in der Schweiz in den Jahren seit Inkrafttreten des FZA erstaunlich stabil blieb.

\subsection{Auswirkungen auf die Sozialversicherungen}
Zuwanderung verlangsamt die Alterung der Bev�lkerung und entlastet damit die umlagefinanzierten Sozialversicherungen der ersten Saeule (AHV/IV/EO/EL). Arbeitnehmende aus EU/EFTA Staaten leisten heute deutlich mehr Beitraege an diese Sozialversicherungen, als sie daraus beziehen.\\

\section{Einleitung}
1. Juni 2002 Inkrafttreten des Freizuegigkeitsabkommen zwischen Schweiz und EG.\\
Seit dem 1. Mai 2011 galt die volle Personenfreizuegigkeit auch f�r die EU8-Staaten. (Osterweiterung)

\section{Einfluss FZA auf Migrationsbeweg. \& Bestand der auslaendischen Wohnbevoelkerung}
\subsection{2.2 Auswirkungen des FZA auf die Migration in die und aus der Schweiz}
Am staerksten wurde die Personenfreizuegigkeit von deutschen und portugiesischen Staatsangehoerigen genutzt.\\
Anteil am Wanderungssaldo 2002-2011: Deutschen 47\%, Portugiesen 19\% und Franzosen 11\%

\subsection{2.3 Bedeutung der Zuwanderung f�r das Bevoelkerungswachstum in der Schweiz}
Die Zuwanderung war in den letzten Jahrzehnten stets eine bedeutende Determinante des Bevoelkerungswachstums in der Schweiz.

\subsection{2.4 Entwicklung des Auslaenderbestandes nach Nationalitaetengruppen}
Die gr�sste Auslaendergruppe stellen die italienischen und deutschen Staatsangehoerigen mit einem Anteil von je 16\%, gefolgt von den Portugiesen mit 13\%.

\subsection{2.6 Entwicklung der Zuwanderung in den einzelnen Regionen der Schweiz}
Vor allem wirtschaftliche Zentren wie bspw. die Genferseeregion (Genf und Waadt), Basel, Zug, Z�rich und Genf sowie drei touristisch ausgerichtete Kantone Wallis, Tessin und Graubuenden verzeichneten relativ zur Bevoelkerung ueberdurchschnittliche Zuwanderungsraten.



\section{Auswirkungen des FZA auf den Schweizer Arbeitsmarkt}


\subsection{3.1.4 Ausbildungsniveau und Berufsgruppen der Zuwanderer}
Vergleich von je 8 Jahren von 1986-1994 und 1994-2002: Anteil der erwerbstaetigen Auslaenderinnen und Auslaender mit mindestens einem Sekundarstufe II- Abschluss ist von 51\% auf 71\% und jener mit einem Abschluss auf Tertiaerstufe von 15\% auf 38\% angestiegen.

\subsubsection{Erwerbstaetigkeit nach Nationalitaet und Berufsgruppe}
In den letzten 8 Jahren hatten EU Angehoerige vor allem in Berufsgruppen einen Beschaeftigungszuwachs zu verzeichnen, bei denen auch Schweizerinnen und Schweizer und andere Auslaender die Erwerbstaetigkeit deutlich ausbauten.\\
Gewisser Zuwachs von Erwerbstaetigen aus der EU27/EFTA war in Berufsgruppen zu erkennen, die sich unterdurchschnittlich oder gar ruecklaeufig entwickelten, wie bspw. bei Fachkraeften in der Landwirtschaft, bei Anlagen- und Maschinenbedienern sowie bei Handwerks- und verwandten Berufen.\\
\\
Scheint es, dass die zugewanderten Arbeitskraefte aus der EU in der Mehrzahl eine gute Ergaenzung des ansaessigen Arbeitskraeftepotentials in stark wachsenden Arbeitsmarktsegmenten darstellen.\\
In die gleiche Richtung deutet eine Auswertung der Erwerbslosenquoten:\\
Drei Berufsgruppen mit dem deutlichsten Zuwachs von Erwerbstaetigen aus dem EU27/EFTA Raum (Fuehrungskraefte, akademische Berufe, Techniker/innen und gleichrangige Berufe) wiesen im betrachteten Zeitraum zwischen 2003 und 2011 unterdurchschnittliche und sinkende Erwerbslosenquoten auf.\\
\\
Studien und diese Daten kamen zum Schluss, dass zusaetzliche auslaendische Arbeitskraefte nicht zu einem Rueckzug der Einheimischen vom Arbeitsmarkt fuehren, sondern diese auf dem Arbeitsmarkt in aller Regel ergaenzen.


\subsection{3.1.5 Erwerbstaetigkeit von EU/EFTA-Staatsangehoerigen nach Branchen}
Erwerbstaetigenanteil von Auslaendern aus dem EU27/EFTA Raum war 2011 im Gastgewerbe (33\%) am hoechsten.\\
Baugewerbe (29\%), verarbeitendes Gewerbe (27\%), Immobilien und sonstige wirtschaftliche Dienstleistungen(25\%)\\
Besonders starke Zunahmen verzeichneten in den letzten acht Jahren die Bereiche Information und Kommunikation sowie das Kredit und Versicherungsgewerbe\\
\\
Insgesamt kann man damit festhalten, dass sich die Zuwanderung aus dem EU-Raum zwar auf relativ spezifische Berufs- und Qualifikationsgruppen, jedoch nicht so sehr auf spezifische Branchen konzentrierte.

\section{Studien zum Thema}
Favre, Sandro (2011), "The Impact of Immigration on the Wage Distribution in Switzerland", NRN: The Austrian Center for Labor Economics and the Analysis of the Welfare State, Working Paper 1108, August 2011.\\
Gerfin, Michael \& Boris Kaiser (2010), "The Effects of Immigration on Wages: An Application of the Structural Skill-Cell Approach", in: Schweizerische Zeitschrift f�r Volkswirtschaft und Statistik, Vol. 146, No. 4, S. 709-739.








\end{document}